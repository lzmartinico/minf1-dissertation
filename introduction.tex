%TODO: expand
The \textit{Chatting about data} project aims to explore the potentials of using chatbot technologies to empower users by increasing their understanding of their own health through data. The project is split between different areas of human health: sleep, mental health, and nutrition. The latter provides an interesting variety of challenges that have been explored in academic and clinical trials, but never for the purpose of a comprehensive, general purpose consumer application. This might be an explanation of why it’s still not common to discuss your dietary habits with your favourite personal assistant, regardless of how ubiquitous the technology has become in the last decade. But while the trend of consumer voice assistants has been developing products as all-round helpful tools, which do not specialise in any one task, the lack of a clear winner in the space of textual digital assistants has resulted in the proliferation of programs with smaller scope. Known as chatbots, they usually specialize in accomplishing a small set of specific tasks, akin to the app model of the early smartphone era, and are similarly distributed through a centralized platform. Thus, just like food logging, a facet of the growing quantified self movement, has exploded with the smartphone, it also falls well within the scope of existing chatbots.
And just as the smartphone has not completely solved food logging, with adoption still relatively low compared to tracking other vitals and abandonment being high, the relatively unexplored area of natural language assistants in the context of \textit{m-health} (mobile health) still presents several unsolved problems.
A nutritional assistant must contain two essential features: the storage of foods consumed by the user, and the retrieval of such information for the purposes of informing dietary choices. There is a large variation in how the two are accomplished among various existing solutions, and the aim of our project is proposing an implementation that, using a Natural Language interface and the ubiquity and capabilities of modern messaging platforms, makes interactions easier and more engaging. \\
We describe our process in creating the Healthbot chatbot on Facebook Messenger, from designing its functionality, to implementing its logic, and running a user evaluation to compare it to \textit{MyFitenssPal}, one of the most popular food logging consumer technology. We also explain the various challenges we have encountered during our implementation, and how they relate to the wider field of chatbot design, as well as the insight we gained from our participants. \\
Our contributions to the field are the first integration of textual and photographic food logging in a chatbot interface, as well as a comparative experiment between the chatbot and traditional diet tracking app on subjects from a similar demographic group. %First relative value
