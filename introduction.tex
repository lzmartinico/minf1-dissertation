The \textit{Chatting about data} project aims to explore the potential of using chatbot technologies to empower users by increasing their understanding of their own health through data. The project explores different areas of human health: sleep, mental health, and nutrition. The latter provides an interesting variety of challenges that have been explored in academic and clinical trials, but never for the purpose of a comprehensive, general purpose consumer application. This may explain why it’s still not common to discuss your dietary habits with your favourite personal assistant, regardless of how ubiquitous the technology has become in the last decade. But while the trend of consumer voice assistants has been developing products as all-round helpful tools, which do not specialise in any one task, the lack of a clear winner in the space of textual digital assistants has resulted in the proliferation of programs with smaller scope.\\
Known as \textit{chatbots}, they usually specialize in accomplishing a small set of specific tasks, akin to the app model of the early smartphone era, and are similarly distributed through a centralized platform. This means meal tracking, a facet of the growing \textit{Quantified Self} movement that has exploded with the smartphone, might well be suitable for the scope of operations of chatbots. And just as the smartphone has not completely solved food logging, with adoption still relatively low compared to tracking of other vitals and abandonment being high, the relatively unexplored area of natural language assistants in the context of \textit{m-health} (mobile health) still presents several unsolved problems.

A nutritional assistant must contain two essential features: the storage of foods consumed by the user, and the retrieval of such information for the purposes of informing dietary choices. There is much variation in how these two tasks are accomplished among various existing solutions, and the aim of our project is proposing an implementation that, using a Natural Language interface and the ubiquity and capabilities of modern messaging platforms, makes interactions easier and more engaging.

Automated food logging in most commercial solutions is predominantly text based, with the user typing in the name of what they had on the day, selecting the closest match from a database of foods and their nutritional value, and estimating the amount of portions they consumed. It has also become popular to add an option to scan the barcodes that are present on all commercial food packaging, linked to their database entries. Some experiments have been conducted to estimate calorie content through image recognition, but there are few commercial products available to the general public, and most experimental implementations require a marker for dimension, which is a significant detriment to usability. \\
We propose implementing a hybrid textual and photographic input food assistant, where portion size can be specified by typing in a quantity, or by characterizing consumption in comparison to previous meals. This eliminates the cognitive load of using a complicated interface or trying to remember precise portion sizes, and will prompt the user to start reflecting on how much they have been eating recently, rather than focus on number of calories tracked. We hope that thinking about past nutritional history in relative terms and exposing overall trends could be more useful than rigorous data collection for the purposes of maintaining long term engagement.

The harder problem is information retrieval, or rather how to extract value from the data and display it in a useful manner. The traditional techniques include calorie counting and breakdown of nutritional information based on macro and micro nutrients, usually shown as a table, a graph, or a progress bar. These strategies have been disputed by the nutrition community, and do not seem to be effective on the general public, resulting in poor engagement and little understanding of underlying dietary issues. \\
Our solution involves handcrafting a small set of simple heuristics to detect harmful dietary patterns, and initiating a conversation by highlighting the issue and giving some suggestions to the user. While this method is unsatisfactory and "does not scale" to deal with the whole gamut of possible dietary problems, there are so far no better solutions in the literature, and we hope having a natural language dialogue will be more accessible to a non-expert than having to interpret the traditional markers in other fitness apps. 

We describe our process in creating the Healthbot chatbot on Facebook Messenger, from designing its functionality to implementing its logic, and running a user evaluation comparing it to \textit{MyFitnessPal}, one of the most popular meal tracking consumer technologies. We also explain the various challenges we have encountered during our implementation, and how they relate to the wider field of chatbot design, as well as the insight we gained from our participants. Chapter 2 provides some background information on the state of the fields of chatbot development and its relationship with the Quantified Self movement and nutrition tracking in particular. Chapter 3 describes the design and architecture of Healthbot, along with a review of features that we originally set to implement, and why it was impossible to complete them. Chapter 4 shows the steps we have taken to review the chatbot's functionality, both during development and after, describing our evaluation study and participants' responses. Chapter 5 elaborates on the results gathered from the survey and observation of how participants interacted with the prototype, and identifies future problems that will have to be tackled before distribution to the public.

Our contributions to the field are implementing and deploying the first integration of textual and photographic food logging in a chatbot interface, as well as a comparative experiment between the chatbot and traditional diet tracking app on subjects from a similar demographic group. Our chatbot is also the first to utilise relative units of measurement for capturing logs, a simplification that has been well-received by testers and shows some promise for future implementations.
