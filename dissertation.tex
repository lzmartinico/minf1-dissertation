\documentclass[bsc,frontabs,twoside,singlespacing,parskip,deptreport]{infthesis}

\begin{document}

\title{This is the Project Title}

\author{Your Name}

\course{Master of Informatics}
\project{{\bf MInf Project (Part 1) Report}}

\date{\today}

\abstract{
This is an example of {\tt infthesis} style.
The file {\tt skeleton.tex} generates this document and can be 
used to get a ``skeleton'' for your thesis.
The abstract should summarise your report and fit in the space on the 
first page.
%
You may, of course, use any other software to write your report,
as long as you follow the same style. That means: producing a title
page as given here, and including a table of contents and bibliography.
}

\maketitle

\section*{Acknowledgements}
Acknowledgements go here. 

\tableofcontents

%\pagenumbering{arabic}


\chapter{Introduction}

The document structure should include:
\begin{itemize}
\item
The title page  in the format used above.
\item
An optional acknowledgements page.
\item
The table of contents.
\item
The report text divided into chapters as appropriate.
\item
The bibliography.
\end{itemize}

Commands for generating the title page appear in the skeleton file and
are self explanatory.
The file also includes commands to choose your report type (project
report, thesis or dissertation) and degree.
These will be placed in the appropriate place in the title page. 

The default behaviour of the documentclass is to produce documents typeset in
12 point.  Regardless of the formatting system you use, 
it is recommended that you submit your thesis printed (or copied) 
double sided.

The report should be printed single-spaced.
It should be 30 to 60 pages long, and preferably no shorter than 20 pages.
Appendices are in addition to this and you should place detail
here which may be too much or not strictly necessary when reading the relevant section.

\section{Using Sections}

Divide your chapters into sub-parts as appropriate.

\section{Citations}

Note that citations 
%%(like \cite{P1} or \cite{P2})
can be generated using {\tt BibTeX} or by using the
{\tt thebibliography} environment. This makes sure that the
table of contents includes an entry for the bibliography.
Of course you may use any other method as well.

\section{Options}

There are various documentclass options, see the documentation.  Here we are
using an option ({\tt bsc} or {\tt minf}) to choose the degree type, plus:
\begin{itemize}
\item {\tt frontabs} (recommended) to put the abstract on the front page;
\item {\tt twoside} (recommended) to format for two-sided printing, with
  each chapter starting on a right-hand page;
\item {\tt singlespacing} (required) for single-spaced formating; and
\item {\tt parskip} (a matter of taste) which alters the paragraph formatting so that
paragraphs are separated by a vertical space, and there is no
indentation at the start of each paragraph.
\end{itemize}

\chapter{Introduction}
%TODO: expand
The \textit{Chatting about data} project aims to explore the potentials of using chatbot technologies to empower users by increasing their understanding of their own health through data. The project explores different areas of human health: sleep, mental health, and nutrition. The latter provides an interesting variety of challenges that have been explored in academic and clinical trials, but never for the purpose of a comprehensive, general purpose consumer application. This might be an explanation of why it’s still not common to discuss your dietary habits with your favourite personal assistant, regardless of how ubiquitous the technology has become in the last decade. But while the trend of consumer voice assistants has been developing products as all-round helpful tools, which do not specialise in any one task, the lack of a clear winner in the space of textual digital assistants has resulted in the proliferation of programs with smaller scope. Known as chatbots, they usually specialize in accomplishing a small set of specific tasks, akin to the app model of the early smartphone era, and are similarly distributed through a centralized platform. Which means food logging, a facet of the growing quantified self movement that has exploded with the smartphone, might well be suitable for the scope of operations of chatbots. And just as the smartphone has not completely solved food logging, with adoption still relatively low compared to tracking other vitals and abandonment being high, the relatively unexplored area of natural language assistants in the context of \textit{m-health} (mobile health) still presents several unsolved problems. \\
A nutritional assistant must contain two essential features: the storage of foods consumed by the user, and the retrieval of such information for the purposes of informing dietary choices. There is a large variation in how these two tasks are accomplished among various existing solutions, and the aim of our project is proposing an implementation that, using a Natural Language interface and the ubiquity and capabilities of modern messaging platforms, makes interactions easier and more engaging. \\
Automated food logging in most commercial solutions is predominantly text based, with the user typing in the name of what they had on the day, selecting the closest match from a database of foods and their nutritional value, and estimating the amount of portions they consumed. It has also become popular to add an option to scan the barcodes that are present on all commercial food packaging, linked to their database entries. Some experiments have been conducted to estimate calorie content through image recognition, but there are few commercial products available to the general public, and most experimental implementations require a marker for dimension, which is a significant detriment to usability. \\
We propose implementing a hybrid textual and photographic input food assistant, where portion size can be specified by typing in a quantity, or by characterizing consumption in comparison to previous meals. This eliminates the cognitive load of using a complicated interface or trying to remember precise portion sizes, and will initiate a first user reflection on how much they have been eating recently, rather than focusing on number of calories tracked, based on the belief that relative quantities and overall trends could be more useful than rigorous data collection for the purposes of maintaining long term engagement. \\
The harder problem is information retrieval, or rather how to extract value from the data and display it in a useful manner. The traditional techniques include calorie counting and breakdown of nutritional information based on macro and micro nutrients, usually shown as a table, a graph, or a progress bar. These strategies have been disputed by the nutrition community, and do not seem to be effective on the general public, resulting in poor engagement and little understanding of underlying dietary issues. \\
Our solution involves handcrafting a small set of simple heuristics to detect harmful dietary patterns, and initiate a conversation by highlighting the issue and giving some suggestions to the user. While this method is unsatisfactory and ``does not scale'' to deal with the whole gamut of possible dietary problems, there are so far no better solutions in the literature, and we hope having a natural language dialogue would be more accessible to a non-expert than having to interpret the traditional markers in other fitness apps. \\

We describe our process in creating the Healthbot chatbot on Facebook Messenger, from designing its functionality, to implementing its logic, and running a user evaluation to compare it to \textit{MyFitnessPal}, one of the most popular food logging consumer technology. We also explain the various challenges we have encountered during our implementation, and how they relate to the wider field of chatbot design, as well as the insight we gained from our participants. Chapter 2 provides some background information on the state of the fields of chatbot development and its relationship with the quantified self movement and nutrition tracking in particular. Chapter 3 describes the design and architecture of Healthbot, along with a review of features that we originally set to implement, and why it was impossible to complete them. Chapter 4 shows the steps we have taken to review the chatbot's functionality, both during development and after, describing our evaluation study and participants' responses. Chapter 5 elaborates on the results gathered from the survey and observation of how participants interacted with the prototype, and identifies future problems that will have to be tackled before distribution to the public.\\
Our contributions to the field are implementing and deploying the first integration of textual and photographic food logging in a chatbot interface, as well as a comparative experiment between the chatbot and traditional diet tracking app on subjects from a similar demographic group. Our chatbot is also the first to utilise relative units of measurment for capturing logs, a simplification that has been well received from testers and shows some promise for future implementations.

\chapter{Background}
\section{The chatbot revolution}
Much has been said about how the rapid reduction in cost of the semiconductor has changed the world in a significant way in the last 60 years. The rapid spread of inexpensive and energy efficient computers, networks, and storage facilities, has revolutionised the way we access information, exchange goods and services, and communicate with one another. The diffusion of the smartphone, specifically, has brought forth an explosion in the amount of information generated globally, with more than $\frac{2}{3}$ of the world population owning one \cite{wearesocial}. Adults in the United Kingdom spend an average of 2 hours per day interacting with their phones, browsing the web, using applications, generating tracking data, and chatting, the latter being one of the most popular applications, with 42\% of mobile users \cite{mobilesocial} being active on social media. The top downloaded messaging applications, as of January 2018, are \textit{Messenger} and \textit{WhatsApp} (1.3 billion active users each), both owned by \textit{Facebook}, and \textit{WeChat} (980 million), owned by China-based \textit{Tencent} \cite{mobilestatista}. \\
Besides keeping up with friends and professional contacts, business transactions are also conducted through chat, either arranging sale and delivery of goods, or for customer assistance, the former being much more prevalent in Asia, where most medium and large size companies, as well as some smaller ones, have a WeChat presence and conduct what is known as \textit{conversational commerce}. Increasingly, many of these transactions are being automated through the deployment of \textit{chatbots} (bots), an evolution of classic conversational interfaces that have become popular in the last decade for commercial and entertainment applications \cite{Dale2016}. The most popular bot platform outside China is Facebook Messenger, which introduced the functionality to developers in April 2016 \cite{Messenger2016}, and it has since taken off with more than 200,000 bots on the platform as of December 2017 \cite{Messenger2017}. The development of Voice Assistants such as Google Assistant, Siri, Cortana, Alexa, or open-sourced Mycroft, have also pushed the deployment and popularity of conversational interfaces, either through voice or textual input. But if the above mentioned technologies try to be a general purpose digital assistant, most chatbots are typically concerned with a smaller domain problem, such as booking flights, checking the status of a bus, or telling a simple story \cite{meisel}. \\
The recent uptick in chatbot usage can not be attributed only to marketing, but also to significant advances in Natural Language Processing (NLP) and Natural Language Understanding (NLU). 
While early chatbot implementations relied on simple pattern matching rules based on recognition of specific words (entity recognition) or parts-of-speech (POS tagging), most of today's chatbot frameworks can leverage large corpora to apply machine learning algorithms, such as Intent analysis. Conversation can follow a slot, or a flow model: the latter is a hardcoded scripted flow diagram that guides the user through a preset conversation; the former specifies ``slots'' that contain some data the developer is interested in, and the chatbot will use NLP techniques to fill the slots from conversations with the user. Responses are typically pre-written and matched to an intent, but advances in deep learning are opening up possibilities for generative models, which create the answer from scratch \cite{Gregori}. Particularly successful can be combinations of several approaches, such as Serban, 2017's use of reinforcement learning to combine the approach of a generative deep learning model and a template-based retrieval model \cite{Serban2017}. Critical to the success of the chatbot is a good context management system, to ensure that a multi-turn conversation doesn't feel disjointed, and that previously entered information remains available to the chatbot throughout the session. All of this functionality is implemented by a growing variety of open source and commercial tools available today \cite{JavierCouto}.\\
From a service provider's perspective, the potential of using a chatbot instead of a human to provide customer service or present content can be very appealing, offering the opportunity of automating repetitive tasks. A successful example of a chatbot deployed in a classroom setting is Georgia Tech's \textit{Jill Watson}, which complemented a team of Teaching Assistants to answer questions on a high traffic class forum \cite{Eicher2016}. But if chatbots can decrease the load for a smaller team, and improve customer experience by decreasing response times, they can also cause increases in customer dissatisfaction. In fact, conversational skills and friendliness are important elements when interacting with a customer service representative \cite{Kang2013}. Within these interactions, a particular emphasis on tone should be given \cite{morris1988many}, an aspect of conversation that until recently had not been developed in bots \cite{Hu2018}. While a complete replacement is still far off, and would cause concern for the livelihood of a still large number of people, chatbots are likely to soon be adopted by more companies to deal with clients' initial queries, while having a human agent supervise the conversation and be ready to intervene when necessary. The centralization of services under a single interface might also address the phenomenon of ``app fatigue'': smartphone owners are no longer installing new apps, and when they do retention rates are abysmal \cite{appfatigue}. While giving up apps for chatbots might limit some kinds of deeper integration into users' devices, it also significantly increases the company's reach to everyone who is registered on a messaging platform, rather than the few who will choose to install new software. \\ 
Chatbot users report their main motivation to be the increase in productivity in information retrieval tasks, compared to installing an app, scanning a long webpage, or placing a phone call, as well as the possibility of receiving customised replies based on their own preferences \cite{10.1007/978-3-319-70284-1_30}. The decrease in attention spans in younger generations \cite{Wilmer2017} and addictive design might explain why young users of social media crave immediate feedback \cite{brandtzaeg2016should}. Thus, as a synchronous form of communication, chat might be perceived to increase productivity over an asynchronous medium like email, and is reflected in users' favouring conciseness in the chatbot's personality \cite{10.1007/978-3-319-67744-6_28}. Other reasons for people to use chatbots, to a lesser extent, are the entertainment value, the social aspect of conversation, and the novelty value, with users actively selecting chatbots as a tool because they fulfil some form of psychological gratification, according to the use and gratification theory \cite{10.1007/978-3-319-70284-1_30}. \\
Given the need of chatbots to be used productively, user needs will cause significant consequences for the field of Human-Computer Interaction (HCI): new paradigms of interface design will have to be invented, and novel approaches to combine different types of outputs will be possible \cite{Følstad2017}. Undoubtedly, conversational interfaces provide a great advantage to the less tech-savy, who might have trouble understanding the user interfaces of many bespoke applications, but will already be familiar to the ``unifying'' chat window, a philosophy promoted as Zero UI \cite{zeroui}. The blank canvas offered by the chat interface offers an unlimited potential for User Interface Design, to the point that, if the chatbot application is aware of enough information about its users, it might be able to create personalised interaction taylored to their preferences. Doing so may alleviate the growing digital divide that some segments of the population experience, because of the implicit biases of tech companies employees when designing user experience without considering the tech-illiterate \cite{Brandtzaeg2011}.
\section{Advances in mHealth}
Since the earliest days of chatbots, such as Eliza, which was modelled after a Rogerian psychotherapist \cite{weizenbaum1966eliza}, it is clear that conversational agents can have significant impact on health. Their applications, however, remained limited, for once because of the restriction of having to use a teletype, but mostly because the still primitive state of the technology made freeform chatting, and, crucially, the ability to understand anything about mental health, impossible. More recently, the development of cheap sensors and widespread connectivity through smartphones has spurned a growing sector of m-health applications. From the 2000s, mobile phone use made patient - doctor communication quicker and more frequent, as well as enabling some initial forms of monitoring and providing a hub for Body Area Network, including different medical sensors \cite{Patrick2008}. The 2010s have seen further developments along these uses, as well as the advance of "Artificial Intelligent" applications, provided by vast quantities of data collected through mobile devices, as well as advances of other Computer Science and Engineering disciplines, such as image recognition, virtual reality, robotics, drones, 3D-printing, and the Internet of Things (IoT) being applied to the medical field \cite{Pistorius2017}. The vast quantities of data being collected have helped to advance the state of the art on several medicinal application - but they also provide a valuable monetisable resource, both from the perspective of Big Data analysis, and for the number of for-profit advertising companies that will sell medical information to marketers and insurance companies \cite{tanner2016}. And whenever personal data is put online, it will also attract hackers, who might be interested in patient data for its resale value on the black market and for identity theft \cite{hackercare}. The sensitive nature of this kind of information makes it difficult to operate without breaking patient confidentiality, since anonymised medical records can be re-identified by correlating with outside sources \cite{Sweeney2001}, and even large experienced medical institution and data collection companies can breach existing regulations \cite{deepmindnhs}. \\
Smartphone and wearable devices users have also been collecting their own personal information, giving birth to the idea of the Quantified Self, or Personal Informatics. \cite{rapp20014}
Much of the Quantified Self movement is based on the idea that the automation of data gathering will lead into greater insight and improve our own health and behaviours by making us reflect and evaluate our past experiences. Data collection for Quantified Self purposes in activity and sleep tracking has boomed in the last decade, with the proliferation of inexpensive inertial measurement units and heartrate monitor sensors combined in wristworn factors, as well as step counting applications being bundled in many smartphones \cite{Crawford2015}.
Rivera and Pelayo, 2012 \cite{Rivera-Pelayo2012} propose a framework necessary for a self tracking app, based on the three activities of tracking cues, triggering and recall.
Tracking can be done through software logging or hardware sensors. Triggering can be active, when the user is prompted to reflect in a suitable context, or passive, where the information is simply displayed in a location the user can observe and notice significant changes. Recalling can be aided through different techniques, which usually involve a considerable amount of post-processing and enhanced by access to large datasets. There is still much work that could be done in the area of contextualization, by associating the collected data with other sources, and data fusion, comparing your own data with independent self or peer reporting, as well as better data visualization in attractive and intuitive ways. \\
One example where the quantified self phenomenon can have a real impact is preventive medicine, promoting healthy lifestyle to alleviate future medical issues. Diet, in particular, has been shown to benefit from open ended food-logging more than other methodologies \cite{Bingham1994}. While Turner-McGrievy, 2013 \cite{Turner-McGrievy2013} found little advantage in replacing a paper diet tracking system with a mobile application, Personal Activity trackers in their study did receive an advantage; this leads to speculation that the lack of improvement in the first group might have been caused by the diet tracker's UX. In fact, using the My Meal Mate app over a 6-month trial, Carter, 2013 \cite{carter2013adherence} reported increased adherence, usage, convenience, social usability, and overall satisfaction compared to traditional diet tracking. \\
Good examples of currently active commercial quantified self apps for fitness and nutrition are MyFitnessPal \cite{mfpwebsite} and Google Fit \cite{googlefitwebsite}, whose designs have been shown \cite{Suzianti2017} to prioritize continuance intention (the willingness to continue using the app), usability qualities (directness, informativeness, learnability, efficiency and simplicity), and user value features (satisfaction, customer need, attachment, pleasure and sociability). The usability of these interfaces, which include both a textual input and a barcode scanner for commercial products, has increased the number of DIY food loggers \cite{Alonso2015}; however, there is still some friction to a seamless logging experience \cite{Boushey2016}. The output of fitness tracking app is also of questionable usefulness: using calories as a primary metric is an oversimplification, as different kind of foods provide more nutritional value for the same caloric amount \cite{webmdcalories}. This might mislead a user in consuming more food than they should be, just to hit their calorie goal, or to leave some essential nutrient off their table, which can have negative consequence on overall health and impact their weight goals. In fact, Hebden, 2014 \cite{hebden2014} reports that user engagement with mHealth food logging solutions tapered off after a month of sustained use. Within the study, patients were most engaged by the text messaging component of the food logging system: this suggests that some of the issues with current interfaces might be alleviated by the use of a chatbot. \\
Chatbots have been speculated to provide a useful tool as a behavioural intervention technology, used to complement human practitioners in reaching a larger number of users and automate personalised messages \cite{Gabrielli2017}. Experimental and clinical trials using simpler informational chatbots have been made in various medical fields, such as counselling \cite{Cameron}, mental health intervention \cite{Elmasri2012} and sexual health information \cite{Brixey2017}, generally providing positive results, or at least giving indication that the medium might be used to address the specific issues.
Among the more successful experiments in Behavioural Health Intervention, the MobileCoach open source platform \cite{mobilecoacheu} was originally designed as a text messaging based system, which did some parsing on the backend but mostly relied on practitioner's interventions. It has now been redesigned as a full fledged online chat platform, which was perceived positively in clinical trials where participants treated for obesity interacted with a chatbot that exhibited a distinct personality \cite{Kowatsch2017}. \\
One successful attempt at making a dietary tracking chatbot is Forksy \cite{forksywebsite}, which seems to be still actively developed, but there are no statistics on its usage and success rates. From direct experience, Forksy is very aggressive with its reminder to log your meals, and does a nice job displaying your diary, but seems to have no ``smart'' features, and is inconsistent with the quality of its parsing.
\section{Making a smart chatbot application}
As researchers in many other fields in Computer Science are realising in the last decade, the collection of large quantities of data can have other uses besides record keeping, by leveraging the booming fields of machine learning and data science. This could eventually lead to make a truly useful chatbot, which in the future could replace or complement professional dietitians in a larger measure than today's solutions. \\
Ever since Richards, 1902 \cite{Richards1902a}, attempts have been made to algorithmically use food composition values to maximise food value per money spent. However, as Richards herself notes, "we know too little of the effect on digestibility, of cooking, and of the combination of two or more foods in one dish, or at one meal, to permit of very close calculation". \\
Even a century later, nutritional science still struggles to establish criteria to categorise individual food as ``good'' or ``bad'', because of the large number of nutrients each is made up of \cite{USDAFoodandNutritionService2007}. While dietary guidelines exist based on nutrients rather than foods, they often fail to be effective, because of the difficulty for consumers to find options that combine all nutritional recommendation, while also being economically accessible, convenient, and conforming to their taste and cultural preferences \cite{Green2015}. \\
To achieve a truly smart dietary assistant, given a vast amount of information about users, their habits and goals, we should be able to recommend an effective strategy to achieve the latter by analysing their choices in the former. As Gregori, 2017 \cite{Gregori} describes, the architecture of a chatbot requires four components: a frontend, a knowledge base, a backend and a corpus. While there are many tools that can be used for chatbot frontend and backend, finding an appropriate corpus and generating a knowledge base are domain specific tasks, with far less options available. \\
Even for a restricted dietary task such as reducing fats consumption, current expert systems are based off a set of handcrafted rules \cite{Prochaska2005}. While the medical community has made efforts to solidify their field into knowledge bases, there are no prevailing standards to read and interpret them, and although some attempts have been made to use knowledge graph representation to power a symptoms identifier chatbot \cite{minutoloa2017conversational}, there doesn't seem to be a canonical dietary knowledge base. Current commercial apps use a combined approach of total calorie counts and macro/micro nutrient percentages, but this is often insufficient to initiate healthy behaviours \cite{Davis2016}. \\
Despite the criticism for the occasional sensationalism, the emerging field of dietary epidemiology advocates a holistic approach to nutrition studies, by taking into account genetic, lifestyle and metabolic information as much as dietary records, making the mere tracking insufficient to draw anything but the most casual inferences on the users' health \cite{byers2001food}. But until this branch of the field develops enough to provide us with effective personalised nutrition (some recent startups would like us to believe that's already the case \cite{habitwebsite}), it's possible to use a more restricted approach based on recognising unbalanced diets from the lack or excess of certain key nutrients, abstracting the mechanics of quantifying exact measures from the users by providing more immediate advice through food recommendation. Data analysis techniques on food composition can be used to draw networks of complementary foods (foods that together fulfil nutritional needs) \cite{Kim2015a}, which could be used to give suggestions based on what users have already eaten. There are plenty of choices for nutritional value composition datasets (the most popular compiled from the United States Department of Agriculture \cite{usda}), and free or commercial APIs \cite{foodapis}.\\
%% social component
Success in activity tracking is influenced by demographics, with older and lower income subjects having lower rates of initial activation and retention \cite{Patel2017}. This problem may be caused by the bespoke user interface each fitness tracker comes with, which we believe a universal chat interface will alleviate. However, there are other factors worth considering for integration. \\
Popular fitness tracking apps often providing social networking functionalities, which have helped participants achieve their fitness goal through a combination of competition with their peers and social accountability \cite{chenchen2014}. Gamification has also proven useful \cite{doi:10.1001/jamainternmed.2017.3458}, and so have financial involvement, but only when profiled as a loss and not for modest gains \cite{doi:10.7326/M15-1635}. One company who successfully integrated diet tracking with all these aspects (gamification, monetary incentives and social accountability) was Gym-Pact (later Pact app), which rewarded users for tracking their calories, eating enough fruits and vegetables, and exercising, but took money from them if they didn't, and allowed users to post their progress to social media and compare it with their peers. The app reached a sufficiently large number of participants \cite{nudgingpracticioner} to sustain itself for several years, and a high percentage of users was frequently able to achieve their goals without cheating thanks to progress reviews from other users (but their business model was not profitable enough, and they closed in Summer 2017).
%% ease of use of image recognition
\section{Image recognition}
Most messaging apps today come with media functionality integrated; in particular, it's easy to take pictures and send them as a message from within the app itself. A diet tracking chatbot might benefit from the users' ability to take pictures and instantaneously receive feedback on their nutritional value. While, to the best of our knowledge, this has never been attempted within a chatbot interface, photographic diet diary have complemented food logging for many years, both in a traditional paper form to aid recollection \cite{Higgins2009}, and, more recently, electronically. For most of the photographic food logging app, portion size estimation requires placing a fiducial marker, an object with a distinctively recognisable pattern, in the frame of the image, to enable fitting a geometrical model on the entire picture \cite{Ahmad2016}. A slight twist on this has been given by Smartplate, a startup that uses a distinctively shaped plate to implement image based food tracking \cite{smartplate}.\\
A different approach was used by Google research with the Im2Calories Android app \cite{Myers2015}. Besides using a convolutional network based off a newly collected multilabel dataset to classify what the food in the picture is, different CNNs are also used to segment images and to estimate their 3D volumetry. This allows the app to assign calorie counts to images that contain different foods in the same plate, and to have more precise estimation of size. Unfortunately, neither the app nor the datasets have been publicly released. More recently, small startups like Calorie Mama\cite{caloriemamaai} and Bitesnap\cite{bitesnap}, as well as Samsung's digital assistant Bixby \cite{bixbyarticle}, have also implemented similar functionality, although it is still not clear how their models were trained or how effective they will be. \\
Among social media users, especially on the Instagram photo sharing platform, it's common to photograph images of aesthetically pleasing food. While this does by no mean provide an exhaustive nutritional history, it can be used as a further automation to save users from having to manually log their meals and extract nutritional information, as well as another potential avenue to establish social accountability to log healthy food \cite{Sharma:2015:MCN:2740908.2742754}. We will still have to use computer vision algorithms on this data, because Instagram tags are unreliable in identifying the content of the picture due to the large number of slang-related false positives \cite{hospedales2016}.
%% issues with image recognition????

\chapter{Evaluation}
As with all software artefacts that have a user facing component, testing can be lead both on a technical level and on a usability level. Since the chatbot infrastructure contains many different components, it will be necessary to conduct some testing on the robustness of each.
\section{Testing}
Throughout the whole implementation, particular care was taken to follow software engineering best practices. Because JavaScript is a dynamically typed language, we lack a compiler's check for code correctness. As a consequence, we have to run code to find out if it works. To avoid deploying a broken commit to the server, we wrote a git pre-commit hook to test if the program doesn't crash immediately, and to pass a linter to catch any syntax error in the code:
\begin{lstlisting}
npm start > /dev/null &
sleep 5
if ``eslint *.js`` && [[ -n `pidof -k node` ]] ; then
    echo "Pass linter and npm doesn't crash"
    exit 0
else
    pkill node
    exit 1
fi
\end{lstlisting}
While this was a good tool to statically catch errors, some bugs would only emerge through dynamic testing. While the \textit{testmybot} unit test library \cite{} looked like a promising solution for establishing a routine of test-driven development, it soon was evident that the Facebook hooks for the underlying Botium library \cite{} are still not mature enough to be used in production. Therefore, rather than having a collection of sample conversations we could feed \textit{testmybot} to determine whether any new code change would break any of the responses we had been getting before, we had to resort to manually testing each new feature, by messaging the chatbot from a personal Facebook account, repeating the same script for each different functionality we had previously implemented. Most debugging information was printed to the Heroku server logs through the \textit{console.log} JavaScript function.
\section{Evaluation}
For our evaluation, we ran an experiment giving out the chatbot to 11 university students, all within ages of 20 to 25 and at least moderately physically active, to use for a week. As a control group, another 9 university students were prescribed to use the MyFitnessPal app for the same duration. All users were recruited through Facebook chat or in person, and all were given the OK to start the evaluation on the same day after having read and signed a consent form describing the experiment and the tester's role in it. In retrospect, having a more gradual rollout might have helped with spotting the first bugs sooner, and giving us a chance to fix the underlying issues without compromising the platform for every other user. As it was, while we identified several issues and features that would have been immediately easy to add, we did not push most of the modification to avoid breaking existing users' workflows. Unfortunately Facebook does not allow to have a separate testing and production environments until the application goes through a first review process, which we couldn't afford to spend time going through.
\subsection{Record keeping}
Since this was our first usage of the chatbot outside our own testing, we expected to encounter a variety of bugs and phrasings that it had never encountered from us. We set up a detailed logging function for all error case, printing the user ID as well so as to be able to reconstruct the causes at a later stage. We could also access a complete record of all communication through the Facebook app console, as well as having a list of intents identified and how the parameters were matched from the Dialogflow agent. While having this much access gave us some great insight into what might be affecting faulty behaviours, it was also concerning how we could read the conversations in their entirety, and while Dialogflow allows to deactivate the logging, there was no way of doing that through Facebook. And even if there was, it would be trivially easy to still log everything through the server.


\subsection{Experiment description - survey, in person interview}
To initiate the experiment, the chatbot users' Facebook profiles were added as testers through the Facebook developer console. They were then sent a link to the chatbot's Facebook page, where they could press a clearly visible button to start chatting. This would open a chat window, where they had the option of pressing a button to get started before being taken through their first conversation. Users were given no indication on how to precede, except for the chatbot's introductory message. Over the course of the evaluation, users sent us some questions (never through the chatbot) on what they could do with it. \\
The MyFitnessPal testers were asked to give feedback a week after the evaluation started; the bot testers were sent feedback forms after 9 days.
The surveys sent to both testers were built using the Google Forms online tool. Most questions were similar to both questionnaires, with some variation when it came to input methods and displays dependent on the app. In compliance with the consent form, none of the questions were made compulsory, and the survey was made anonymous.
The questionnaire asked some background information on the participant, to establish levels of fitness and computer literacy, thoughts on nutrition and previous dietary and food tracking histories.
Testers were then asked their opinion on the usability, utility, pleasantness of the entire platform they were evaluating, as well as for each specific functionality, and if they had any feedback on things they would have liked to see. Some answers were multiple choices, checkboxes or Likert scales, but most were open text input to allow the participant to give a full explanation of the reasoning behind their answer
\subsubsection{Survey Response}
7 participants responded to the My Fitness Pal survey within the first day, and 9 to the chatbot specific survey. While some questions were answered by all the participants who took the survey, none of the open ended questions were answered by all, sometimes with around half the respondents ignoring one question (??Verify??). \\
Participants seemed to be distributed similarly across the two trials, with chatbot users being slightly more proficient with computers, as well as being more aware of their fitness levels. Similar splits were evident in the proportions of participants who had died before, with around three quarters of participants citing a good current health or scepticism with established diet, and some chatbot users using laziness as a motivation. The minority of users who had dieted chose to do so because of environmental, athletic or health-related issues, but did not maintain dieting after reaching their goals, or because of commitment issues. Among both groups, about half the participants consistently had 3 meals per day, with some having a variable number of meals and no participants consuming less than two; our chatbot users however in general snacked less than My Fitness Pal users (there's a possibility that these answers might have been influenced by the experiment, even if participants were encouraged to think about their behaviour before; the fact that My Fitness Pal presents snack as a distinctly separate category, and the chatbot doesn't, might have affected responses to this question). \\
About half of the participants reported having tracked their diet before, either keeping a food diary, memorizing their meal, or, the majority of respondents, using My Fitness Pal, and the majority of previous trackers also kept a record of their snacks. \\
For unclear reasons, more than half the respondents skipped the section about their dietary makeup, but for who did fill it in, definitions of ``balanced diet'' varied significantly: while a majority named a variation of having a correct proportion of Protein, Carbohydrates and Fats, with some allowing for vitamins and minerals as well, others named calories as a main concern, reducing some unhealthy food groups and increasing others, or avoiding stressing about their diet and making sure to have what made them feel good. Only half of the respondents consider their diet to be balanced, including all those who planned their meals in advanced, and most respondents tend to cook their own meals, eat out or do both things in equal measure. \\
My Fitness Pal user found the app on average more useful than chatbot user, although the latter was rated as generally more pleasant to use. The food diary, the macronutrient breakdown graph and the remaining calorie counter were all generally considered clear and useful, with the graph being the most pleasant feedback. For input, the majority of users preferred scanning the barcode of the meal they were having, although for some the kind of food they were eating was a factor, and people who tended to cook their own food preferred text entries. All users had some issues with finding the food they wanted using text entry, but no one complained about the method being too slow; barcode scanning seemed to perform better, with only some users reporting difficulties identifying a barcode or matching the correct item in the database. By contrast, chatbot users generally found the feedback useless, or insufficient. Chatting was highly preferred as an input method, although several participants didn't find it understood their queries well enough, and some were annoyed by the prompts for size. Some users who took pictures for input found it wasn't accurate, but the larger problem for the feature seemed to be people who weren't aware of the functionality. \\
Retention rates were much higher for My Fitness Pal users, with the largest missed meals estimated to be 5, and some user logging all their meals; chatbot users, instead, were much less active, with one person logging almost every meal, and everyone else estimating having missed between 5 and 20. In both cases, the leading cause of missing a meal log was lack of time or forgetfulness, with some chatbot users finding input methods cumbersome or lack of interest because the feedback didn't seem useful. As a consequence, almost all chatbot testers did not log their meals on several days. Half of the users report having received a reminder the day after, with the other saying they didn't (which might be explainable by the fact that, while everyone received a reminder, it wasn't necessarily after a day of inactivity). The reminders were generally found to be useful, and mostly made the users log their food on the day, and one user even expressed a desire to receive more frequent prompts to avoid forgetting more meals. My Fitness Pal also provided a reminder functionality, but it's off by default. All users who turned it on got a reminder, but it didn't make them use the app after. \\
One stark difference in response between chatbot and app user was on preference between noting their food records with absolute measurement (number of portions or unit of measurement plus numerical value). My Fitness Pal users overwhelmingly declared a preference for absolute value metrics, because of the need to calculate precise calorie counts that the app provides, and as a more reliable comparison method to standard recommended portion sizes. The majority of chatbot users instead indicated a preference for relative values, because it's easier not to have to constantly measure portions. \\
Despite the fact that the utility of a food diary comes from the ability to look back on previous meals, only a third of the chatbot users, and just over half the app users took advantage of this feature at a later date, and those who did reported the information presented to them to be accurate, but unhelpful; in fact, about half of the chatbot users and two thirds of the app user don't think using the meal log has given them a better idea of how they eat. \\
Overall, most participants did not think that logging their food had helped them to eat better, although for many users that was because they already are happy with their diet. Those that registered a positive impact mentioned that having a better oversight on their food trends did prove helpful for them, and My Fitness Pal user specified sugar tracking and suggested recipes as useful features, although some comments also pointed out that the paid version of the app could have been more useful. However, $\frac{2}{3}$ of chatbot users found that they had become more ``mindful'' about their diet by using the chatbot, as opposed to less than half of the My Fitness Pal users. \\
Expectations for the chatbot were high for some users who were hoping for \textit{[a] good AI} which would be talkative and give them active reminders and regular feedback; some were just looking for a more convenient way to log their food; but most participants did not expect much from it. Needless to say, the former group were disappointed by our implementation, with the natural language parsing of quantities, repetitive replies and image recognition capabilities being particularly frustrating. At the same time, users appreciated the ability to choose input method, and some found the chatbot's personality less annoying than they expected. My Fitness Pal tester also were expecting ease of use, a complete database, and a tool that would prompt small change in their behaviours by highlighting trends that were needed to be changed. Most of these were met by participants, although the majority of American commercial products in the database was deemed a problem. \\
When asked if they would continue to log their meals after the evaluation period, participants on both platforms were mostly uninterested, either because they didn't find it useful enough, or because logging took too much time, and in the case of the chatbot, they perceived the product development as not being ready enough for regular usage. However, some users who seemed to have benefited from its usage were willing to continue using, or at least consider it in case of future more rigorous dieting, and one My Fitness Pal user was convinced to resume their paper food diary. \\
About half the My Fitness Pal users enabled fitness tracking functions, which seemed generally well received, although there were some concerns to how accurate their estimations were, and how useful it is to simply subtract exercise from calorie intake from a nutritional standpoint. Participants who did not use the feature were potentially interested, but the interface wasn't easy to understand, and there were perceived barriers to entry such as downloading a separate companion app or owning a smartwatch to better track calorie expenditure.\\
Testers of the app suggested they would have liked to have dedicated fruit and vegetable counters, automatic exercise calorie calculations and personalized recipe suggestions based on a specific ingredient or past meals and goals. For the chatbot, suggestions included pointing out a food's recommended amount, more reminders, especially around 5-a-day tracking, retroactively adding past meals, adding more variation to the automatic replies to make them less boring, and better onboarding functionality.
As part of the survey participants were  also asked if they thought that the information they were uploading was being kept safe, and if they thought it was an important concern. Most participants were actually concerned about their dietary records being exposed, with some particularly concerned with being judged because of their diet, while others didn't think food records were a particularly sensitive topic, and anonymising dietary data could be used to benefit medical research organisations. Users of the chatbot generally considered their information to be secured, and while one participant specified ``I know its developer takes security seriously'', another identified that platform issues were a problem ``I mean it's on facebook so not really.''. On the other hand, My Fitness Pal users were more concerned or unsure whether their information was safe or not, and with good reason: two days after the study completed, the app's parent company \textit{Under Armour} publicly announced it had been a victim in one of the largest ever leaks of user personal information \cite{underarmour}.

% use the following and \cite{} as above if you use BibTeX
% otherwise generate bibtem entries
% consider \nocite{*}
\bibliographystyle{plain}
\bibliography{bibliography.bib}

\end{document}
