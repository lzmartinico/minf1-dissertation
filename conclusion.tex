The main goal of our project was creating an alternative interface to make it easier for health data to be collected and understood. \\
Our implementation had us explore the architecture of popular SaaS toolkit Dialogflow and its functionalities, working with a platform that's only been superficially documented to create a fully functioning agent. We evaluated some basic principles of chatbot dialogue design, various backend technologies to finally choose Heroku, an option that would grant us the most flexibility and ease of use; we also explored the space of various 3rd party tools, such as databsae technologies, nutritional information databases, image recognition for food objects and social networking APIs. While our original design included a vast breadth of features, having to deal with time limitations forced us to take into consideration what aspects of the interface would be truly fundamental, and prioritise what would take us to a minimum viable products. We highlighted some issues we run into along the way, from failing to obain permission to use the Instagram API for our purposes, to collecting a sufficiently large dataset of food names. Particular effort was put into classifying food based on its nutritional value, but we failed to achieve even barely usable performance, due to the complexity of the problem domain.\\
Despite the many roadblocks, we delivered a functioning prototype, the first chatbot, to our knowledge, which combines texting and pictures as input for a dietary log. We also led the first comparitive user trial between a food logging app and chatbot, with a small group of tester across a single demographic. While our final release was still plagued by many bugs that affected usability, we were able to provide a useful service to some users who gained some insights on their diet. We also achieved better results than MyFitnessPal with the control group in some areas, such as reminders and general pleasantness of use, and found out some interesting correlations between platform used and importance attributed to measurement units, as well as users' mental models and attitudes towards the confidentiality of nutritional data. \\
This leaves us with some promising expectations for future progress of the project, and we explored some possible extentions of various implementation difficulties. Although these features might improve how much users can do with our chatbot, and how it feels to interact with it, we will still have to find a definitive way to replace our ad-hoc heuristic with a fully scalable and complete knowledge base, which would require a more systematic approach in its design, as well as larger scale testing. As widely as these areas could be explored in the next phase of our project, however, during our development phase, and from feedback we received during our evaluation, we realised that it would be impossible for a chatbot that deals with medical data without compromising user's trust. Therefore, we would like to redirect our enquiries into exploring the possibility of having a secure chatbot, either from an architectural point of view, or in relation to their compliance with the upcoming General Data Protection Regulation in Europe. We realise that seems to go against recent trends in industry, where increasing data collection to enable intelligent features has become the norm, but we recognise that chatbots can create an illusion of intimacy that might lead to sensitive information being shared more willingly, and users have a right to have their most personal data protected.
