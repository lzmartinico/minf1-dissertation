\section{The chatbot revolution}
Much has been said about how the rapid reduction in cost of the semiconductor has, in the last 60 years, changed the world in a significant way. The rapid spread of inexpensive and energy efficient computers, networks, and storage facilities, has revolutionised how we access information, exchange goods and services, and communicate with one another. The diffusion of the smartphone, specifically, has brought forth an explosion in the amount of information generated globally, with more than 4.9 billions users, with adults in the United Kingdom spending an average of 2 hours per day interacting with their phones, browsing the web, using applications, generating tracking data, and chatting. The latter usage in particular is one of the most popular, with 42\% of mobile users \cite{mobilesocial} being active social media users. The top downloaded messaging applications, as of January 2018, are Messenger and Whatsapp (1.3 billion active users each) and Wechat (980 million) \cite{mobilestatista}. \\
Besides keeping up with friends and professional contacts, business transactions are also conducted through chat, either arranging sale and delivery of goods, or for customer assistance, the former being much more prevalent in Asia, where most medium and large size companies, as well as some smaller ones, have a WeChat presence and conduct what is known as \textit{conversational commerce}. Increasingly, many of these transactions are being automated through the deployment of \textit{chatbots} (bots), an evolution of classic conversational interfaces that have become popular in the last decade for commercial and entertainment applications \cite{Dale2016}. The most popular bot platform outside China is Facebook Messenger, which introduced the functionality to developers in April 2016 \cite{Messenger2016}, and it has since taken off with more than 200,000 bots on the platform as of December 2017 \cite{Messenger2017}. The development of Voice Assistants such as Google Assistant, Siri, Cortana, Alexa, or open-sourced Mycroft, have also pushed the deployment of conversational interfaces, and to some extent through the chat medium, with all the mentioned agents providing some form of textual input. \\
Besides the marketing pushes, the use of chatbots has increased thanks to advances in Natural Language Processing (NLP) and Natural Language Understanding (NLU). 
While early chatbot implementations relied on simple pattern matching rules based on recognition of specific words (entity recognition) or parts-of-speech (POS tagging), most of today's chatbot frameworks can leverage large corpora to apply machine learning algorithms, such as Intent analysis. Conversation can follow a slot, or a flow model: the latter is a hardcoded scripted flow diagram that guides the user through a preset conversation; the former specifies ``slots'' that contain some data the developer is interested in, and the chatbot will use NLP techniques to fill the slots from conversations with the user. Responses are typically pre-written and matched to an intent, but advances in deep learning are opening up possibilities for generative models, which create the answer from scratch \cite{Gregori}. Particularly successful can be combinations of several approaches, such as Serban, 2017's use of reinforcement learning to combine the approach of a generative deep learning model and a template-based retrieval model \cite{Serban2017}. Critical to the success of the chatbot is a good context management system, to ensure that a multi-turn conversation doesn't feel disjointed, and that previously entered information remains available to the chatbot throughout the session. All of this functionality is implemented by a growing variety of open source and commercial tools available today \cite{JavierCouto}.\\
From a service provider's perspective, the advantage of using a chatbot instead of a human to provide customer service or present content is clear: over the long term, the cost of development is small compared to the number of salaries that would have to be paid to maintain the same amount of concurrent conversations. The centralization of services under a single interface to same extents also addresses the phenomena of ``app fatigue'': smartphone owners are no longer installing new apps, and when they do retention rates are abysmal \cite{appfatigue}. Users' main motivations is that using chatbots makes them more productive compared to going through an app or long webpage to find information, as well as the possibility to received customised replies based on their own interests to a larger extent than other technologies \cite{10.1007/978-3-319-70284-1_30}. This might be a symptom of increasingly shorter attention spans in younger generations \cite{Wilmer2017}, which also explain why a synchronous form of communication such as chat might be perceived to increase productivity compared to an asynchronous medium such as email, and is reflected in users' preference on the chatbot's personality \cite{10.1007/978-3-319-67744-6_28}. To a lesser extent, people also use chatbots for entertainment value and because they benefit from the social aspect, but some of the interest might only be attributed to the novelty value. \\
Given the need of chatbots to be used productively, user needs will cause significant consequences for the field of Human-Computer Interaction (HCI): new paradigms of interface design will have to be invented, and novel approaches to combine different types of outputs will be possible \cite{Følstad2017}. Undoubtedly, it provides a great advantage to the less tech-savy, who might have trouble understanding the user interfaces of many bespoke applications, but will already be familiar to the ``unifying'' chat window, to the point that, if the chatbot application is aware of enough information about its users, it might be able to create personalised interaction taylored to their preferences, and by doing that alleviate the growing digital divide that some segments of the population experience because of the implicit biases of some tech workers \cite{Brandtzaeg2011}.

\section{Advances in mHealth}
Since the earliest days of chatbots, such as Eliza, which was modelled after a Rogerian psychotherapist \cite{weizenbaum1966eliza}, it's been clear that conversational agents can have significant impact on health. Their applications however remained limited, for once because of the restriction at the time of having to use a teletype, but mostly because the still primitive state of the technology made freeform chatting, and crucially the ability to understand anything about your mental health, impossible. More recently, the development of cheap sensors and widespread connectivity through smartphones has spurned a growing sector of m-health applications. From the 2000s, mobile phone use made patient - doctor communication quicker and more frequent, as well as enabling some initial forms of monitoring and providing a hub for Body Area Network including different medical sensors \cite{Patrick2008}. The 2010s have seen further developments along these uses, as well as the advance of "Artificial Intelligent" applications, provided by vast quantities of data collected through mobile devices, as well as advances of other Computer Science and Engineering disciplines such as image recognition, virtual reality, robotics, drones, 3D-printing, and the Internet of Things (IoT) being applied to the medical field \cite{Pistorius2017}. The vast quantities of data being collected have helped to advance the state of the art on several medicinal application - but they also provide a valuable monetisable resource, both from the perspective of Big Data analysis, and for the number of for-profit advertising companies that will sell medical information to marketers and insurance companies \cite{tanner2016}, as well as hackers for resale on the black market and identity theft \cite{hackercare}. The sensitive nature of this kind of information makes it difficult to operate without breaking patient confidentiality, since even if anonymised medical records can be re-identified by correlating with outside sources \cite{Sweeney2001}, and even large experienced medical institution and data collection companies can breach existing regulations \cite{deepmindnhs}. \\
Smartphone and wearable devices users have also been collecting their own personal information, giving birth to the idea of the Quantified Self, or Personal Informatics. 
Much of the Quantified Self movement is based on the idea that the automation of data gathering will lead into greater insight and improve our own health and behaviours by reflecting and evaluating our past experiences. Rivera and Pelayo, 2012 \cite{Rivera-Pelayo2012} propose a framework necessary for self tracking app, based on the three activities of tracking cues, triggering and recall.
Tracking can be done through software logging or hardware sensors. Triggering can be active, when the user is prompted to reflect in a suitable context, or passive, where the information is simply displayed in a location the user can observe and notice significant changes. Recalling can be aided through different techniques, which usually involve a considerable amount of post-processing and enhanced by access to large datasets. There is still much work that could be done in the area of contextualization, by associating the collected data with other sources, and data fusion, comparing your own data with independent self or peer reporting, as well as better data visualization in attractive and intuitive ways. \\
One example where the quantified self phenomenon can have a real impact is preventive medicine, promoting healthy lifestyle to alleviate future medical issues. Diet, in particular, has been shown to benefit from open ended food-logging more than other methodologies \cite{Bingham1994}. While Turner-McGrievy, 2013 \cite{Turner-McGrievy2013} found little advantage in replacing a paper diet tracking system with a mobile application, Personal Activity trackers in their study did receive an advantage; this leads to speculation that the lack of improvement in the first group might have been caused by the diet tracker's UX; and in fact, using the My Meal Mate app over a 6-month trial, Carter, 2013 \cite{carter2013adherence} reported increased adherence, usage, convenience, social usability, and overall satisfaction compared to traditional diet tracking. \\
Good examples of currently active commercial quantified self apps for fitness and nutrition are \textit{MyFitnessPal} \cite{mfpwebsite} and Google Fit \cite{googlefitwebsite}, whose designs have been shown \cite{Suzianti2017} to prioritize continuance intention (the willingness to continue using the app), usability qualities such as directness, informativeness, learnability, efficiency and simplicity; user value features such as satisfaction, customer need, attachment, pleasure and sociability. The usability of these interfaces, which include both a textual input and a barcode scanner for commercial products, has increased the number of DIY food loggers \cite{Alonso2015}; however, there is still some friction to a seamless logging experience \cite{Boushey2016}, which might be bridged by the use of a chatbot interface.  \\
Chatbots have been speculated to provide a useful tool as a behavioural intervention technology, used to complement human practitioners in reaching a larger number of users and automate personalised messages \cite{Gabrielli2017}. Experimental and clinical trials using simpler informational chatbots have been made in various medical fields, such as counselling \cite{Cameron}, mental health intervention \cite{Elmasri2012} and sexual health information \cite{Brixey2017}, generally providing positive results, or at least giving indication that the medium might be used to address the specific issues.
Among the more successful experiments in Behavioural Health Intervention, the MobileCoach open source platform \cite{mobilecoacheu} was originally designed as a text messaging based system, which did some parsing on the backend but mostly relied on practitioner's interventions. It has now been redesigned as a full fledged online chat platform, which was perceived positively in clinical trials where participants treated for obesity interacted with a chatbot that exhibited a distinct personality \cite{Kowatsch2017}. \\
One successful attempt at making a dietary tracking chatbot was Forksy \cite{forksywebsite}, which seems to be still actively developed, but there are no statistics on its usage and success rates. Forksy is very aggressive on getting users to log their diet and does a nice job displaying content, but seems to have no ``smart'' features.
\section{Making a smart chatbot application}
As researchers in many other fields in Computer Science are realising in the last decade, the collection of large quantities of data can have other uses besides record keeping, by leveraging the booming fields of machine learning and data science, and could eventually lead to make a truly useful chatbot to replace or complement professional dietitians in a larger measure. \\
Ever since Richards, 1902 \cite{Richards1902a}, attempts have been made to algorithmically use food composition values to maximise food value per money spent. However, as Richards herself notes, "we know too little of the effect on digestibility, of cooking, and of the combination of two or more foods in one dish, or at one meal, to permit of very close calculation". \\
Even a century later, nutritional science still struggles to establish criteria to categorise any one food as ``good'' or ``bad'', because of the large number of nutrients that make up each food \cite{USDAFoodandNutritionService2007} \\
To achieve a truly smart dietary assistant, we should be able, given a vast amount of information about our users, their habits and goals, to recommend an effective strategy to achieve the latter by analysing their choices in the former. As Gregori, 2017 \cite{Gregori} describes, the architecture of a chatbot requires four components: a frontend, a knowledge base, a backend and a corpus. While there are many tools that can be used for chatbot frontend and backend, finding an appropriate corpus and generating a knowledge base are domain specific tasks, with far less options available. \\
Even for a restricted dietary task such as reducing fats consumption, expert systems will be based off a set of handcrafted rules \cite{Prochaska2005}. While the medical community has made efforts to solidify their field into knowledge bases, there are no prevailing standards to read and interpret them, and although some attempts have been made to use knowledge graph representation to power a symptoms identifier chatbot \cite{minutoloa2017conversational}, there doesn't seem to be a canonical dietary knowledge base. Current commercial apps use a combined approach of total calorie counts and macro/micro nutrient percentages, but this approach is often insufficient to initiate healthy behaviours \cite{Davis2016}.
Despite the criticism for the occasional sensationalism, the emerging field of dietary epidemiology advocates a holistic approach to nutrition studies, by taking into account genetic, lifestyle and metabolic information as much as dietary records, making the mere tracking in sufficient to draw anything but the most casual inferences on the users' health \cite{byers2001food}. But until this branch of the field develops enough to provide us with effective personalised nutrition (some recent startups would like us to believe that's already the case \cite{habitwebsite}), it's possible to use a more restricted approach based on recognising unbalanced diets from the lack or excess of certain key nutrients, abstracting the mechanics of quantifying exact measures from the users by providing more immediate advice through food recommendation. Data analysis techniques on food composition can be used to draw networks of complementary foods (foods that together fulfil nutritional needs) \cite{Kim2015a}, which could be used to give suggestions based on what users have already eaten. There are plenty of choices for nutritional value composition datasets\cite{usda}, and free or commercial APIs \cite{foodapis}\\
%% social component
Success in activity tracking is influenced by demographics, with older and lower income subjects having lower rates of initial activation and retention \cite{Patel2017}. This problem may be caused by the bespoke user interface each fitness tracker comes with, a problem which might be alleviated by using a universal chat interface.\\
Popular fitness tracking apps often providing social networking functionalities, which have helped participants achieve their fitness goal through a combination of competition with their peers and social accountability \cite{chenchen2014}. Gamification has also proven useful \cite{doi:10.1001/jamainternmed.2017.3458}, and so have financial involvement, but only when profiled as a loss and not for modest gains \cite{doi:10.7326/M15-1635}. One company who successfully integrated diet tracking with monetary incentives and social accountability was Gym-Pact (later Pact app), which rewarded users for tracking their calories, eating enough fruits and vegetables, and exercising, but took money from them if they didn't. The app reached a sufficiently large number of participants \cite{nudgingpracticioner} to sustain itself for several years, and a high percentage of users was frequently able to achieve their goals (but their business model wasn't profitable enough, and they closed in Summer 2017).
%% ease of use of image recognition
\section{Image recognition}
Most messaging apps today come with media functionality integrated; in particular, it's easy to take pictures and send them as a message from within the app itself. A diet tracking chatbot might benefit from the users' ability to take pictures and instantaneously receive feedback on their nutritional value. While this as to the best of our knowledge never been attempted within a chatbot interface, photographic diet diary have complemented food logging for many years, both in a traditional paper form to aid recollection \cite{Higgins2009}, and more recently electronically. Classically, portion size estimation required placing a fiducial marker, an object with a distinctively recognisable pattern, in the frame of the image, as to be able to fit a geometrical model on the entire picture \cite{Ahmad2016}. A slight twist on this has been spun by Smartplate, a startup that uses a distinctively shaped plate to implement image based food tracking \cite{smartplate}.\\
A different approach was used by Google research with the Im2Calories Android app \cite{Myers2015}. Besides using a convolutional network based off a newly collected MultiLabel dataset to classify what the food in the picture is, different CNNs are also used to segment images and to estimate their 3D volumetry. This allows the app to assign calorie counts to images that contain different foods in the same plate, and to have more precise estimation of size. Unfortunately, neither the app nor the datasets have been publicly released. More recently, small startups like Calorie Mama\cite{caloriemamaai} and Bitesnap\cite{bitesnap}, as well as Samsung's digital assistant Bixby \cite{bixbyarticle} have also implemented similar functionality, although it's still not clear how their models were trained, or how effective they will be. \\
Among social media users, especially on the Instagram photo sharing platform, it's common to photograph images of aesthetically pleasing food. While this does by no mean provide an exhaustive nutritional history, it can be used as a further automation to save users from having to manually log their meals and extract nutritional information, as well as potentially another avenue to establish social accountability to log healthy food \cite{Sharma:2015:MCN:2740908.2742754}. We will still have to use computer vision algorithms on this data, because Instagram tags are unreliable in identifying the content of the picture due to the large number of slang-related false positives \cite{hospedales2016}.

%% issues with image recognition????
